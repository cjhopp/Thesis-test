# Introduction
Matched filter earthquake detection uses waveform cross-correlation between continuous seismic data and known earthquake recordings to identify additional events in a seismic catalogue. Correlation-based detection offers improved performance over traditional, amplitude-based techniques due to its ability to detect signals in noisy data and when multiple events are closely spaced in time. This significantly increases the number of events detected without increasing the rate of false detections. These advantages make matched filter detection ideal for monitoring microseismicity in areas of geothermal power generation, which are characterized by numerous noise sources and the possibility for dense clusters of small-magnitude, induced seismic events. Detecting a larger number of earthquakes, specifically including those thought to be induced by fluid injection, can also be useful in characterizing and monitoring the state of the geothermal resource with time. Using an earthquake catalog of 637 events occurring between 1 January and 18 November 2015 as our initial dataset, we implemented a matched filtering routine for the Mighty River Power geothermal fields at Rotokawa and Ngatamariki, North Island, New Zealand. Preliminary runs detected nearly 22,000 additional events across both geothermal fields. On average, each of the 637 template events detected 45 additional events throughout the study period, with a maximum number of additional detections for a single template of 359. We discuss these detections, rates of detection and double difference locations for the detected events in the context of Mighty River Power plant operations and fluid injection activities. 